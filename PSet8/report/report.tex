\documentclass[12pt, a4paper]{article}
\usepackage[a4paper, bindingoffset=0.2in, %
left=0.5in,right=0.5in,top=0.5in,bottom=0.5in,%
footskip=.25in]{geometry}
\usepackage{graphicx}
\usepackage{amssymb}
\usepackage{amsmath}
\usepackage{hyperref}
\usepackage{physics}

\title{PSet8 Report}
\author{Ali Abolhassanzadeh Mahani}


\begin{document}
	\maketitle
	\section{First Order Differential Equation}
	The equation we are to solve is
	\begin{equation}
		R \dv{Q}{t} + \frac{Q}{C} = V
	\end{equation}
	for the values $V= 10 volts$, $R = 3000 \Omega$, $C = 1 \mu F$ and the simulation time from zero to $t = 0.5\, ms$.\\
	Since the numbers are a little overwhelming, we make parameter changes to come up with better numbers for our simulation.
	
	The change of variables is as follows:
	\begin{equation} \label{eq:change}
		\begin{aligned}
			&x \equiv \frac{Q}{R C} - \frac{V}{R}, \quad \tau \equiv \frac{t}{R C}\\
			\Rightarrow  &\dv{x}{\tau} + x = 0
		\end{aligned}
	\end{equation}
	
	Using this change in variables, the time limit becomes: $\tau = \frac{0.5 ms}{10^{-6} F \times 3000 \Omega} = \frac{1}{6} s$
	Also, if we want the capacitor to start from being empty, the initial value for $x$ will be: $x_0 = \frac{1}{300} \frac{volts}{\Omega}$
	
	The analytical solution is obtained by integrating the equation $\frac{\dd{x}}{x} = \dd{\tau}$
	which gives:
	\begin{equation} \label{eq:analytical}
		x = x_0 \exp(- \tau)
	\end{equation}
	\section{$2^{nd}$ Order ODE}
	I made all the functions to take variables \texttt{x\_init, acc, step, time}. \texttt{x\_init} is the initial position of the mass from the origin.
	\texttt{acc} is the function that returns the \emph{acceleration} as a function of position \texttt{x}.
	The initial conditions are as follows:
	\begin{equation}
		\begin{aligned}
			\dot{x} &= 0\\
			x & = 1
		\end{aligned}
	\end{equation}
	
	For some methods like \emph{verlat} I had to define extra initial conditions due to the nature of the algorithms.
	
	I made a dictionary \texttt{data} that stores the results for different methods and makes the job of plotting and other stuff easier.
	I integrated for 60 time units and time step of \texttt{step = 0.01}. Then I plotted them in one graph. (Fig \ref{fig:ODEs})
	\begin{figure}[h!]
		\centering
		\includegraphics[width=0.8\linewidth]{../p2/ode2_plots.jpg}
		\label{fig:ODEs}
		\caption{plot of the solutions using different integration methods from $0$ to $t = 60$ with time step of $0.01$}
	\end{figure}
	Then, I plotted $\dot{x}$ as a function of $x$ for each method separately. (Fig\ref{fig:energy})
	\begin{figure}[h!]
		\centering
		\includegraphics[width=0.45\linewidth]{../p2/euler.jpg}
		\includegraphics[width=0.45\linewidth]{../p2/euler_koomer.jpg}
		\includegraphics[width=0.45\linewidth]{../p2/verlat.jpg}
		\includegraphics[width=0.45\linewidth]{../p2/velocity_verlat.jpg}
		\includegraphics[width=0.45\linewidth]{../p2/beeman.jpg}
		\label{fig:energy}
		\caption{$\dot{x}$ vs. $x$ for different methods.}
	\end{figure}
	
	As can be seen from Fig\ref{fig:energy}, the \emph{verlat}, \emph{velocity verlat}, and \emph{beeman} methods have stability for this
	problem.
	
	\section{Instability in algorithms}
	Here we use the same change in variables that we did in equation\ref{eq:change} to simulate the charging capacitor.
	The algorithm we are going to use here is as follows:
	\begin{equation} \label{eq:instability}
		y_{n + 1} = y_{n - 1} + 2 \dot{y}_n h
	\end{equation}
	where $h$ is the time step. Since this is a two step algorithm, we need a second initial value that we get using the Euler method. From that point
	forward, we use the algorithm in equation\ref{eq:instability}. Then, we plot both the analytical solution (eq.\ref{eq:analytical}) and the 
	numerical solution in one graph to compare them.
	
	In small values for terminating the integration, the numerical solution seems to be stable, but if we wait long enough, we can see that
	from about $\tau = 6.0$, the numerical integration shows signs of periodic motions and oscilation around the analytical solution. 
	(Fig.\ref{fig:instability})
	\begin{figure}[h!]
		\centering
		\includegraphics[width=0.8\linewidth]{../p3/instability.jpg}
		\label{fig:instability}
		\caption{numerical solution using the algorithm from eq\ref{eq:instability}, shows oscilation about the analytical solution for 
			large enough $\tau$, which represents instability for this algorithm in this perticular problem.}
	\end{figure}

	It can also be seen that for higher values of the time step, the instability manifests much sooner and explodes more rapidly.


\end{document}